\documentclass{article}

\usepackage[czech]{babel}

\usepackage[a4paper,top=2cm,bottom=2cm,left=3cm,right=3cm,marginparwidth=1.75cm]{geometry}

% Useful packages
\usepackage{amsmath}
\usepackage{graphicx}
\usepackage[colorlinks=true, allcolors=blue]{hyperref}
\usepackage{listings}
\usepackage{xcolor}
\usepackage{graphicx}

\definecolor{backcolour}{rgb}{0.95,0.95,0.92}
\definecolor{codegreen}{rgb}{0,0.6,0}
\definecolor{codegray}{rgb}{0.5,0.5,0.5}
\definecolor{codepurple}{rgb}{0.58,0,0.82}
\definecolor{backcolour}{rgb}{0.95,0.95,0.92}
\definecolor{codeyellow}{RGB}{255,255,102}

\lstdefinestyle{mystyle}{
    language=Bash,
    backgroundcolor=\color{backcolour},   
    commentstyle=\color{codegreen},
    keywordstyle=\color{codepurple},
    numberstyle=\tiny\color{codegray},
    stringstyle=\color{codepurple},
    basicstyle=\footnotesize\ttfamily,
    breakatwhitespace=false,         
    breaklines=true,                 
    captionpos=b,                    
    keepspaces=true,                 
    numbers=left,                    
    numbersep=5pt,                  
    showspaces=false,                
    showstringspaces=false,
    showtabs=false,                  
    tabsize=2,
    frame=single,
    framerule=0pt,
    framexleftmargin=5pt,
    framexrightmargin=5pt,
    framextopmargin=5pt,
    framexbottommargin=5pt,
    rulecolor=\color{backcolour},
    xleftmargin=8pt,
    xrightmargin=0pt,
    % Add these lines for proper inline code highlighting
    escapeinside={(*@}{@*)}, % if you want to add LaTeX within your code
    moredelim=[is][\color{red}\ttfamily]{|}{|}, % for inline code
    extendedchars=true, % allows non-ASCII characters, such as ü, ä, etc.
    literate=%
        {á}{{\'a}}1
        {í}{{\'i}}1
        {é}{{\'e}}1
        {ý}{{\'y}}1
        {ó}{{\'o}}1
        {ú}{{\'u}}1
        {ř}{{\v{r}}}1
        {č}{{\v{c}}}1
        {ě}{{\v{e}}}1
        {š}{{\v{s}}}1
        {ť}{{\v{t}}}1
        {ž}{{\v{z}}}1
        {ů}{{\r{u}}}1
        {Á}{{\'A}}1
        {Í}{{\'I}}1
        {É}{{\'E}}1
        {Ý}{{\'Y}}1
        {Ó}{{\'O}}1
        {Ú}{{\'U}}1
        {Ř}{{\v{R}}}1
        {Č}{{\v{C}}}1
        {Ě}{{\v{E}}}1
        {Š}{{\v{S}}}1
        {Ť}{{\v{T}}}1
        {Ž}{{\v{Z}}}1
        {Ů}{{\r{U}}}1
}


\lstset{style=mystyle}

\title{Poznámky k započtovému testu z Operačních systémů}
\author{Malinda Kryštof}

\begin{document}
\maketitle


\section{Zadání z předchozých let}
\subsection{Zadání 1}
\begin{enumerate}
  \item Vytvořte soubor male.txt, kde bude text psaný malými písmeny. Vytvořte script velka.sh, který změní velikosti písmen v souboru male.txt a uloží do souboru velke.txt
\begin{lstlisting}[caption=Vytvoříme soubor male.txt]
  echo "toto je náhody text do souboru male.txt" > male.txt
\end{lstlisting}
\begin{lstlisting}[caption=Jak by měl vypadat velka.sh]
#!/bin/bash

# Kontrola, zda existuje soubor male.txt
if [ ! -f male.txt ]; then
    echo "Soubor male.txt neexistuje."
    exit 1
fi

# Převod malých písmen na velká a uložení do souboru velke.txt
tr '[:lower:]' '[:upper:]' < male.txt > velke.txt

echo "Převod na velká písmena byl dokončen."
\end{lstlisting}
\begin{lstlisting}
  chmod +x velka.sh # Přidáme execute permission, aby se script dal spustit
  ./velka.sh # Spustíme script
  cat velke.txt # Zkontrolujeme output scriptu
\end{lstlisting}
  \item Vytvořte soubor ceny v tomto tvaru, kde druhý sloupce je oddělen a třetí platy v jednotlivých pracovníků. Napište skript secti.sh, který sečte všechny platy pro pracovníky oddělení 1.
\begin{lstlisting}
  echo "
  Jahoda 1  22010
  Bruna 2 80000
  Zuda  1 12000
  " > ceny.txt
\end{lstlisting}
\begin{lstlisting}[caption=skript secti.sh]
#!/bin/bash

# Kontrola, zda existuje soubor ceny.txt
if [ ! -f ceny.txt ]; then
    echo "Soubor ceny.txt neexistuje."
    exit 1
fi

# Sečtení platů pro pracovníky oddělení 1
suma=0
while read -r jmeno oddeleni plat; do
    if [ "$oddeleni" -eq 1 ]; then
        suma=$((suma + plat))
    fi
done < ceny.txt

echo "Celková suma platů pro pracovníky oddělení 1 je: $suma"
\end{lstlisting}
\item V Linuxu vytvořte skript s názvem adresare, který z aktuálního adresáře, ve kterém jsou adresáře i soubory, vypíše seznam všěch adresářů do souboru
\begin{lstlisting}
  #!/bin/bash

  # Získání seznamu adresářů v aktualní adresáři
  find . type -d > seznam_adresaru.txt

  echo "Seznam adresařů byl uložen do souboru seznam_adresaru.txt"
\end{lstlisting}
\end{enumerate}
\subsection{Zadání z 23.04.2024}
\begin{enumerate}
  \item V textovém souboru jsou \textit{(vytvořte)} pod sebou názvy 3 souborů a jejich různé přípony oddělené tečkou. Pomocí skriptu vypište tyto přípony do externího souboru s názvem externí
\begin{lstlisting}
#!/bin/bash

# Zdrojový soubor
zdroj="soubory.txt"

# Output soubor
output="externi"

pripony=$(awk -F '.' '{print $NF}' $zdroj)
echo $pripony > $output
\end{lstlisting}
  \item V linuxu vytvořte skript s názvem zapis.sh, který zjistí zda se dá do souboru zapisovat. Výsledkem bude text ma obrazovce \textit{"lze zapisovat"} nebo \textit{"nelze zapisovat"}.
    \begin{lstlisting}[caption=zapis.sh musel jsem modifikovat pipe aby se zobrazil ale normálně je to jen ta vertikalní čárka]
#!/bin/bash

# Název testovacího souboru
test_file=$1

# Funkce pro kontrolu zápisových práv
check_write_permission() {
    if ls -l $test_file "|" grep -q -E "\s*w"; then
        echo "lze zapisovat"
    else
        echo "nelze zapisovat"
    fi
}

# Zobrazení práv souboru
echo "Práva souboru $test_file:"
ls -l $test_file

# Kontrola zápisových práv
check_write_permission
\end{lstlisting}
  \item Po ukončení vše odstraňte a použijte příkaz history -c a history -w
\begin{lstlisting}
history -c && history -w 
# Toto smaže historii našeho shellu (bash)
\end{lstlisting}
\end{enumerate}
\section{Věci s přednášek}
\subsection{AWK}
\begin{center}
\begin{tabular}{|c|c|}
\hline
\textbf{Příkaz} & \textbf{Popis} \\
\hline
\texttt{awk '\{print \$1\}'} & Vytiskne první sloupec z každého řádku \\
\hline
\texttt{awk '\$3 > 10'} & Vytiskne řádky, kde třetí sloupec je větší než 10 \\
\hline
\texttt{awk -F':' '\{print \$1, \$3\}'} & Nastaví oddělovač na dvojtečku a vytiskne první a třetí sloupec \\
\hline
\texttt{awk 'NR==2\{print \$0\}'} & Vytiskne druhý řádek \\
\hline
\texttt{awk '\$NF == "pattern"'} & Vytiskne řádky, kde poslední sloupec je "pattern" \\
\hline
\end{tabular}
\end{center}

\begin{center}
\begin{tabular}{|c|c|}
\hline
\textbf{Operátor} & \textbf{Popis} \\
\hline
\texttt{/pattern/} & Porovnává aktuální řádek s regulárním výrazem \\
\texttt{==} & Porovnává rovnost \\
\hline
\texttt{!=} & Porovnává nerovnost \\
\hline
\texttt{\^} & Porovnává začátek pole \\
\hline
\texttt{\$} & Porovnává konec pole \\
\hline
\texttt{\$1 ~ /pattern/} & Porovnává první pole s regulárním výrazem \\
\hline
\texttt{\$1 !~ /pattern/} & Negace porovnání prvního pole s regulárním výrazem \\
\hline
\end{tabular}
\end{center}
\subsection{SED}
\begin{center}
\begin{tabular}{|c|c|}
\hline
\textbf{Příkaz} & \textbf{Popis} \\
\hline
\texttt{s/pattern/replace/} & Nahradí první výskyt patternu v každém řádku řetězcem replace \\
\hline
\texttt{s/pattern/replace/g} & Nahradí všechny výskyty patternu v každém řádku řetězcem replace \\
\hline
\texttt{/pattern/d} & Vymaže řádek, který obsahuje pattern \\
\hline
\texttt{/pattern/p} & Vytiskne řádek, který obsahuje pattern \\
\hline
\texttt{1,10d} & Vymaže řádky od prvního do desátého \\
\hline
\texttt{\$d} & Vymaže poslední řádek \\
\hline
\texttt{\$p} & Vytiskne poslední řádek \\
\hline
\texttt{i\textbackslash{}Text before} & Vloží text před řádek \\
\hline
\texttt{a\textbackslash{}Text after} & Vloží text za řádek \\
\hline
\end{tabular}
\end{center}
\subsection{Jak psát regex}
\begin{center}
\begin{tabular}{|c|c|}
\hline
\textbf{Výraz} & \textbf{Popis} \\
\hline
\texttt{.} & Kterýkoli znak \\
\texttt{\^} & Začátek řádku \\
\texttt{\$} & Konec řádku \\
\texttt{*} & Nula nebo více opakování \\
\texttt{+} & Jedno nebo více opakování \\
\texttt{?} & Nula nebo jedno opakování \\
\texttt{[]} & Jakýkoli znak uvedený v hranatých závorkách \\
\texttt{[a-z]} & Jakýkoli znak od 'a' do 'z' \\
\texttt{[\^{}]} & Jakýkoli znak, který není uveden v hranatých závorkách \\
\texttt{\textbackslash{}w} & Jakýkoli alfanumerický znak \\
\texttt{\textbackslash{}W} & Jakýkoli nealfanumerický znak \\
\texttt{\textbackslash{}d} & Jakékoli číslice \\
\texttt{\textbackslash{}D} & Jakýkoli znak, který není číslicí \\
\texttt{\textbackslash{}s} & Jakýkoli bílý znak \\
\texttt{\textbackslash{}S} & Jakýkoli nebílý znak \\
\hline
\end{tabular}
\end{center}
\subsection{Příkazy v Linuxu}
\begin{center}
\begin{tabular}{|l|c|c|}
\hline
\textbf{Příkaz} & \textbf{Popis} & \textbf{Příklad} \\
\hline
\texttt{ls} & Vypíše obsah aktuálního adresáře. & \texttt{ls -l} \\
\hline
\texttt{cd} & Změní aktuální pracovní adresář. & \texttt{cd /home/user} \\
\hline
\texttt{mkdir} & Vytvoří nový adresář. & \texttt{mkdir novy\_adresar} \\
\hline
\texttt{rm} & Smaže soubor nebo adresář. & \texttt{rm soubor.txt} \\
\hline
\texttt{cp} & Zkopíruje soubor nebo adresář. & \texttt{cp soubor.txt /cesta/} \\
\hline
\texttt{mv} & Přesune nebo přejmenuje soubor nebo adresář. & \texttt{mv soubor.txt novy\_soubor.txt} \\
\hline
\texttt{cat} & Vypíše obsah souboru. & \texttt{cat soubor.txt} \\
\hline
\texttt{chmod} & Změní oprávnění souboru. & \texttt{chmod +x skript.sh} \\
\hline
\texttt{pwd} & Zobrazí aktuální pracovní adresář. & \texttt{pwd} \\
\hline
\texttt{rmdir} & Smaže prázdný adresář. & \texttt{rmdir prazdny\_adresar} \\
\hline
\texttt{find} & Vyhledá soubory a adresáře. & \texttt{find /home -name "*.txt"} \\
\hline
\texttt{tail} & Vypíše posledních n řádků souboru. & \texttt{tail -n 10 soubor.txt} \\
\hline
\texttt{head} & Vypíše prvních n řádků souboru. & \texttt{head -n 5 soubor.txt} \\
\hline
\texttt{grep} & Vyhledá řetězec v souboru. & \texttt{grep "pattern" soubor.txt} \\
\hline
\texttt{wget} & Stáhne soubor z internetu. & \texttt{wget url} \\
\hline
\texttt{tar} & Archivuje nebo rozbaluje soubory v tar archivu. & \texttt{tar -czf archiv.tar.gz adresar} \\
\hline
\texttt{df} & Zobrazí informace o dostupném diskovém prostoru. & \texttt{df -h} \\
\hline
\end{tabular}
\end{center}

\subsection{Oprávnění v Linuxu}

\begin{center}
\begin{tabular}{|c|c|c|}
\hline
\textbf{Číselná forma} & \textbf{Symbolická forma} & \textbf{Popis} \\
\hline
0 & --- & Žádné oprávnění \\
\hline
1 & --x & Proveditelné \\
\hline
2 & -w- & Zapisovatelné \\
\hline
3 & -wx & Zapisovatelné a proveditelné \\
\hline
4 & r-- & Čitelné \\
\hline
5 & r-x & Čitelné a proveditelné \\
\hline
6 & rw- & Čitelné a zapisovatelné \\
\hline
7 & rwx & Čitelné, zapisovatelné a proveditelné \\
\hline
\end{tabular}
\end{center}

\section{Cvičení 1}
\subsection{Vytvořte nového uživatele s názvem vašeho loginu a nastavte mu heslo: Password1*}
Přepneme se do sudo
\begin{lstlisting}[language=bash]
su -i
\end{lstlisting}
Vytvoříme nového uživatele
\begin{lstlisting}[language=bash]
useradd {vas_login}
\end{lstlisting}
A přidáme požadované heslo Password1*
\begin{lstlisting}[language=bash]
passwd {vas_login}
\end{lstlisting}

\subsection{Přihlaste se jako vas-login. V adresari /home/login vytvořte pět souborů s těmito právy}
\begin{itemize}
    \item text1 rwx r-x rwx
    \item text2 r- - rwx -w
    \item text3 rwx rwx rwx
    \item text4 r-x - -x r- -
    \item text5.. rwx - -x rwx
\end{itemize}

Příhlásíme se
\begin{lstlisting}[language=bash]
su {vas_login}
\end{lstlisting}
Musíme se přepnout do /home/{login}
\begin{lstlisting}[language=bash]
cd $HOME
\end{lstlisting}
\textbf{nebo můžeme použít příkaz}
\begin{lstlisting}[language=bash]
cd /home/{vas_login}
\end{lstlisting}
Vytvoříme pět souboru
\begin{lstlisting}[language=bash]
chmod 751 text1
chmod 704 text2
chmod 777 text3
chmod 511 text4
chmod 762 text5
\end{lstlisting}
\subsection{V domovském adresáři uživatele vas login vytvořte adresář pokus s těmito právy rwxr--r--}
Vytvořte adresář
\begin{lstlisting}[language=bash]
mkdir pokus
\end{lstlisting}
Nastavíme práva
\begin{lstlisting}[language=bash]
chmod 744 pokus
\end{lstlisting}
Zkopírujeme soubor text1 do složky pokus a přepneme se do složky pokus
\begin{lstlisting}[language=bash]
cp text1 pokus/
cd pokus/
\end{lstlisting}
Nastavíme požadovné práva
\begin{lstlisting}[language=bash]
chmod 644 ../pokus
\end{lstlisting}
\subsection{Změna vlastníka souboru text5 na root}
\begin{lstlisting}[language=bash]
sudo chown root text5
\end{lstlisting}
\subsection{Vypsání celé cesty aktuálního adresáře}
\begin{lstlisting}[language=bash]
pwd
\end{lstlisting}
\subsection{Vytvoření symbolického linku k souboru text3 a ověření, že je symbolický}
\begin{lstlisting}[language=bash]
ln -s text3 odkaznatext3
ls -l odkaznatext3
\end{lstlisting}
\subsection{Výpis souborů v adresáři /etc podle zadaných kritérií}
\begin{lstlisting}[language=bash]
cd /etc
ls a*
ls m*f
ls m*b*m*c*f
\end{lstlisting}
\subsection{Spuštění programu mc a zjištění čísla jeho procesu}
\begin{lstlisting}[language=bash]
mc
pgrep mc
\end{lstlisting}
\subsection{Zrušení procesu Midnight Commander}
\begin{lstlisting}[language=bash]
kill PID 
#(kde PID je cislo procesu ziskane z predchoziho prikazu)
\end{lstlisting}
\subsection{Vytvoření souboru adresy v domovském adresáři a zadání obsahu}
\begin{lstlisting}[language=bash]
cd ~
\end{lstlisting}
\begin{lstlisting}[language=bash]
cat > adresy
Petr Bily 123451
Petra Cerna 123213
Jan Vlk 151122
Jiri Cila 122321
\end{lstlisting}
A zmáčkneme \textbf{Ctrl + D}
\subsection{Zobrazení obsahu souboru adresy}
\begin{lstlisting}[language=bash]
cat adresy
\end{lstlisting}
\subsection{Výpis řádků souboru obsahujících slovo \textbf{Petr}}
\begin{lstlisting}[language=bash]
grep "Petr Bily" adresy
\end{lstlisting}
\subsection{Výpis řádků souboru, které neobsahují slovo \textbf{Petr}}
\begin{lstlisting}[language=bash]
grep -v Petr adresy
\end{lstlisting}
\subsection{Výpis řádků souboru obsahujících písmeno \textbf{J} a neobsahujících \textbf{V}}
\begin{lstlisting}[language=bash]
grep "J" adresy | grep -v "V"
\end{lstlisting}
\subsection{Změna malých písmen na velká v souboru adresy a uložení do souboru adresy2}
\begin{lstlisting}[language=bash]
tr '[:lower:]' '[:upper:]' < adresy > adresy2
\end{lstlisting}
\section{Cvičení 2}
\lstinputlisting[language=bash]{cv2/script1.sh}
\lstinputlisting[language=bash]{cv2/script2.sh}
\lstinputlisting[language=bash]{cv2/script3.sh}
\lstinputlisting[language=bash]{cv2/script4.sh}
\lstinputlisting[language=bash]{cv2/script5.sh}

\section{Cvičení 3}
\lstinputlisting[language=bash]{cv3/cv3_1.sh}
\lstinputlisting[language=bash]{cv3/cv3_2.sh}
\lstinputlisting[language=bash]{cv3/cv3_3.sh}
\lstinputlisting[language=bash]{cv3/cv3_4.sh}
\lstinputlisting[language=bash]{cv3/cv3_5.sh}
\section{Cvičení 4}
\subsection{Vytvoření souboru výplaty}

\begin{lstlisting}
echo "
Novak                  Jan                   12000                    Jihlava         
Novotna               Eva                   12521                    Brno           
Jahoda                Petr                   22010                    Praha         
Bruna                  Vaclav               8000                      Praha        
Zuda                   David                 12000                    Jihlava
" > platy.txt
\end{lstlisting}

\subsection{Vypsání prvního a třetího sloupce}

\begin{lstlisting}
awk '{print $1, $3}' platy.txt
\end{lstlisting}

\subsection{Výběr řádků obsahujících slovo Jihlava}

\begin{lstlisting}
awk '/Jihlava/' platy.txt
\end{lstlisting}

\subsection{Vypsání prvního, druhého a čtvrtého sloupce pro řádky obsahující slovo Jihlava}

\begin{lstlisting}
awk '/Jihlava/{print $1, $2, $4}' platy.txt
\end{lstlisting}

\subsection{Výběr všech řádků obsahujících v 4. sloupci písmeno P}

\begin{lstlisting}
awk '$4 ~ /P/' platy.txt
\end{lstlisting}

\subsection{Výběr všech řádků obsahujících v 4. sloupci první písmeno P}

\begin{lstlisting}
awk '$4 ~ /^P/' platy.txt
\end{lstlisting}

\subsection{Výběr všech pracovníků, kteří mají plat 12000}

\begin{lstlisting}
awk '$3 == 12000' platy.txt
\end{lstlisting}

\subsection{Výběr všech pracovníků, kteří mají plat menší než 12000}

\begin{lstlisting}
awk '$3 < 12000' platy.txt
\end{lstlisting}

\subsection{Výběr všech pracovníků, kteří mají plat menší než 20000 a větší než 10000}

\begin{lstlisting}
awk '$3 < 20000 && $3 > 10000' platy.txt
\end{lstlisting}

\subsection{Výběr všech pracovníků, kteří jsou z Brna nebo z Jihlavy}

\begin{lstlisting}
awk '$4 == "Brno" || $4 == "Jihlava"' platy.txt
\end{lstlisting}

\subsection{Vypsání čísla řádku, na kterém je ve sloupci jméno Petr}

\begin{lstlisting}
awk '$2 == "Petr" {print NR}' platy.txt
\end{lstlisting}

\subsection{Zobrazení 2. a 4. řádku}

\begin{lstlisting}
awk 'NR==2 || NR==4' platy.txt
\end{lstlisting}

\subsection{Vypočtení průměrného platu}

\begin{lstlisting}
awk '{sum+=$3} END {print "Průměrný plat: ", sum/NR}' platy.txt
\end{lstlisting}

\subsection{Vypočtení průměrného platu pro všechny pracovníky z Jihlavy}

\begin{lstlisting}
awk '$4 == "Jihlava" {sum+=$3; count++} END {print "Průměrný plat pro pracovníky z Jihlavy: ", sum/count}' platy.txt
\end{lstlisting}

\subsection{Vypsání maximálního platu}

\begin{lstlisting}
awk 'BEGIN {max=0} $3 > max {max=$3} END {print "Maximální plat: ", max}' platy.txt
\end{lstlisting}

\subsection{Vytvoření programu, který každé 2 minuty zkopíruje výpis obsahu vybraného adresáře do souboru}

\begin{lstlisting}[language=Bash]
#!/bin/bash

# Adresář, jehož obsah chceme zkopírovat
source_dir="/cesta/k/vybranemu/adresari"

# Soubor, do kterého chceme zkopírovat výpis obsahu adresáře
target_file="/cesta/k/cilovemu/souboru.txt"

# Zkopírování obsahu adresáře do cílového souboru
ls -l "$source_dir" > "$target_file"

# Zpráva o dokončení kopírování
echo "Výpis obsahu adresáře byl zkopírován do souboru: $target_file"
\end{lstlisting}
\section{Cvičení 5}
\subsection{V systému Linux vytvořte skript s názvem adresare, který vypíše počet adresářů v aktuálním adresáři do souboru}
\lstinputlisting[language=bash, caption=Script adresare.sh]{cv5/adresare.sh}

\subsection{V Linuxu vytvořte adresář cv5 a v něm vytvořte 3 soubory první, druhý, třetí. Dále v něm vytvořte skript s názvem prejmenuj, který dá všem souborům příponu html}
\begin{lstlisting}
  mkdir cv5 #Vytvoří složku
  touch první druhý třetí #vytvoří soubory
\end{lstlisting}
\lstinputlisting[language=bash, caption=Script prejmenuj.sh]{cv5/cv5/prejmenuj.sh}
\subsection{Přihlaste se do Linuxu pod účtem root. Vytvořte skript, který vytvoří nového uživatele a nastaví mu heslo}
\begin{lstlisting}[caption=Přihlasíme se jako root]
  su -i
\end{lstlisting}
\begin{lstlisting}[caption=Můžeme zkontrolovat za koho jsme přihlášení]
  whoami
\end{lstlisting}
\lstinputlisting[caption=Script novy uzivatel.sh]{cv5/novy_uzivatel.sh}
\subsection{V Linuxu vytvořte skript s názvem balíčky, který po spuštění vypíše všechny nainstalované balíčky, které začínají na php. Pokud tam žádný balíček není, nainstalujte ho}
\lstinputlisting[caption=Script balicky.sh]{cv5/balicky.sh}
\subsection{V Linuxu vytvořte skript s názvem existuje, který zjistí , zda je nějaký soubor existuje nebo ne. Výsledkem bude text: „existuje“ nebo „neexistuje“.}
\lstinputlisting[caption=Script existuje.sh]{cv5/existuje.sh}
\section{Cvičení 6}
\subsection{Napište příkaz, který vypíše vaše aktuální UID.}

\begin{lstlisting}
id -u
\end{lstlisting}

\subsection{Napište příkaz, který vypíše seznam skupin, v nichž se nacházíte.}

\begin{lstlisting}
groups
\end{lstlisting}

\subsection{Vytvořte soubor jmena.txt s 21 jmény. Z výstupu jmena.txt vypište jen sudé řádky a  na začátky lichých řádek souboru jmena.txt vložte znak „l", na začátky sudých řádek vložte znak „s".}

\lstinputlisting{cv6/sude_liche.sh}

\subsection{b za každý řádek přidá volný řádek}

\begin{lstlisting}
sed G jmena.txt
\end{lstlisting}

\subsection{Na začátky desáté až dvacáté řádky souboru jmena.txt přidejte znak \#.}

\begin{lstlisting}
sed '10,20 s/^/#/' jmena.txt
\end{lstlisting}

\subsection{Pomocí příkazu najděte soubor jmena.txt v adresáři /home i v podadresářich.}

\begin{lstlisting}
find /home -name "jmena.txt"
\end{lstlisting}

\subsection{Pomocí příkazu najděte v adresáři /home i v podadresarich všechny soubory začínající j s příponou .txt}

\begin{lstlisting}
find /home -type f -name "j*.txt"
\end{lstlisting}

\subsection{Pomocí příkazu najděte v adresáři /home i v podadresářích vše, co bylo změněno před 7 dny.}

\begin{lstlisting}
find /home -type f -mtime +7
\end{lstlisting}

\subsection{Zjištění rodičovského procesu}

\subsubsection{Spusťte zobrazení procesů}

\begin{lstlisting}
ps -l
\end{lstlisting}

\subsubsection{Jaký je rodičovský proces procesu ps?}

Měl by to být uživatelský shell.

\subsection{Zobrazte celou stromovou strukturu všech procesů operačního systému.}

\begin{lstlisting}
pstree
\end{lstlisting}

\subsection{Procesy na pozadí, popředí gedit}

\subsubsection{Spustíme Textový editor GEdit}

\begin{lstlisting}
gedit &
\end{lstlisting}

\subsubsection{Dáme process do pozadí}

\begin{lstlisting}
bg %1
\end{lstlisting}

\subsubsection{Přeneseme proces do popředí}

\begin{lstlisting}
fg %1
\end{lstlisting}

\subsection{Sledujte procesy v reálném čase top}

\begin{lstlisting}
top
\end{lstlisting}

\subsection{Pojmenovaná roura}

\subsubsection{Vytvoříme routu}

\begin{lstlisting}
mkfifo roura
\end{lstlisting}

\subsubsection{Načtení datumu do routy}

\begin{lstlisting}
date > roura &
\end{lstlisting}

\subsubsection{Zobrazení dat}

\begin{lstlisting}
cat roura
\end{lstlisting}

\section{Cvičení 7}
\subsection{Pomocí příkazu vytvořte soubor se jménem „datum". Do tohoto souboru vložte pomocí příkazu aktuální datum.}

\begin{lstlisting}
date > datum
\end{lstlisting}

\subsection{Vypište deset největších souborů a pak deset nejmenších souborů v adresáři /etc}

\subsubsection{Největší soubory}

\begin{lstlisting}
ls -lS /etc | head -n 11
\end{lstlisting}

\subsubsection{Nejmenší soubory}

\begin{lstlisting}
ls -lSr /etc | head -n 11
\end{lstlisting}

\subsection{V adresáři /usr/bin najděte soubor, který byl modifikován naposledy.}

\begin{lstlisting}
ls -lt /usr/bin | head -n 2
\end{lstlisting}

\subsection{Ze souboru studenti.csv vypište jen první sloupec, tj. sloupec s příjmeními (použijte find, případně cut).}

\begin{lstlisting}
cut -d ',' -f 1 studenti.csv
\end{lstlisting}

\subsection{Udělejte kopii souboru studenti.csv, nahraďte pomocí příkazu středníky pomlčkou a pomocí diff se podívejte, které řádky byly změněny.}

\begin{lstlisting}
# Uděláme kopii
cp studenti.csv studenti_copy.csv 

# Nahradíme ;, -
sed -i 's/;/-/g' studenti_copy.csv

# Ukážeme rozdíl
diff studenti.csv studenti_copy.csv
\end{lstlisting}

\subsection{Rozdělte příkazem  studenti.csv soubor na kusy po pěti řádcích (split).}

\begin{lstlisting}
split -l 5 studenti.csv studenti_chunk_
\end{lstlisting}

\subsection{Napište příkaz, který vypíše počet všech (pro vás viditelných) adresářů v podstromu /etc}

\begin{lstlisting}
find /etc -type d | wc -l
\end{lstlisting}

\subsection{V podstromu adresáře /tmp najděte všechny soubory, které jsou větší než sto kilobyte (find).}

\begin{lstlisting}
find /tmp -type f -size +100k
\end{lstlisting}

\subsection{Do své složky (home) zkopírujte soubor /etc/passwd, který přejmenujete například na passwd2. V souboru passwd2 odstraňte nějaké řádky. Vytvořte skript, který projde vytvořené uživatele (/etc/passwd)  a  zjistí, rozdíl se souborem passwd2 (diff). Rozdíl se uloží do souboru rozdil.txt. Tedy zjistí, kteří uživatelé jsou v souboru /etc/passwd navíc.}

\begin{lstlisting}
# Zkopírování souboru /etc/passwd
cp /etc/passwd ~/passwd2

# Odstranímě nějaké řádky (vybral jsem 2 a 5)
sed -i '2d;5d' passwd2

# Spustíme script
\end{lstlisting}
\lstinputlisting[caption=Script který spustíme]{cv7/porovnani.sh}

\subsection{Pomocí příkazu vypište verzi OS linux (číslo jádra a další informace).}

\begin{lstlisting}
uname -a
\end{lstlisting}


\end{document}
